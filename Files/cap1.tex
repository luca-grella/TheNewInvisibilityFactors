\phantomsection
\subsection*{La nostra era: l’apice della \textit{Computer Revolution}}
\addcontentsline{toc}{subsection}{La nostra era: l’apice della \textit{Computer Revolution}}

L’evoluzione ha sempre portato l’uomo a dei bivi. Il dibattito principale su qualsiasi cosa possa portare cambiamento, che sia esso positivo o negativo, è quasi sempre di natura morale: “Ciò che mi fa stare meglio, posso ottenerlo tramite un processo giusto?”, “Cosa è giusto e cosa è sbagliato?”, “Il fine giustifica i mezzi?”, “È eticamente corretto ciò che sto facendo?”, “Anche se ritengo sia sbagliato ciò che sto facendo, è giusto farlo comunque, dato che mi porta beneficio (e/o magari porta beneficio anche alla comunità)?”. Queste sono alcune delle classiche domande che prima o poi tutti ci poniamo nella vita, ma il discorso diventa ancora più interessante se queste domande andassimo ad applicarle, come anticipato prima, all’evoluzione umana. Nello specifico, l’evoluzione umana ha avuto un’accelerazione esponenziale nell’ultimo secolo e la motivazione è chiaramente l’introduzione della tecnologia in ogni ambito. La tecnologia, infatti, ha sempre avuto uno sviluppo ricorsivamente intrinseco; più ci evolviamo tecnologicamente, inventiamo e creiamo nuovi strumenti tecnologici, più lo sviluppo si velocizza, in quanto gli strumenti stessi da noi inventati ci vengono in aiuto. Quindi, maggiore è il numero di tecnologie a nostra disposizione, maggiore sarà la velocità con cui ne creeremo altre. Per questo motivo si può parlare di “accelerazione esponenziale” della nostra evoluzione, che negli ultimi anni è basata solo ed esclusivamente sulla tecnologia.
Questa corsa allo sviluppo tecnologico, però porta a dei cambiamenti rapidissimi non solo nell’approccio alla ricerca del “nuovo” e del “migliore”, ma anche nell’approccio etico.
Il filosofo James H. Moor, da cui ho preso spunto per sviluppare la tesi che discuterò in questo documento, si sofferma, per parlare di etica riguardante lo sviluppo tecnologico, sulla \textit{computer revolution}, e, in particolare, afferma che essa può essere suddivisa in 3 periodi: \textit{introduction stage}, \textit{permeation stage} e \textit{power stage} (Moor 2001). 
Il primo si riferisce al periodo in cui è avvenuta l’introduzione delle tecnologie informatiche e in cui solo poche persone potevano averne accesso. 
Il secondo (durato tra il 1980 e il 2000 circa) è il periodo in cui è avvenuta la permeazione delle tecnologie informatiche all’interno della società e le persone hanno iniziato ad essere sempre più dipendenti da ogni dispositivo elettronico. Gli utenti hanno iniziato a diventare sempre più sofisticati e le tecnologie sempre più \textit{user-friendly}. 
Solo recentemente siamo entrati nel terzo periodo, il \textit{power stage}. Secondo Moor, quello in cui stiamo vivendo è anche l’ultimo periodo della \textit{computer revolution}. Il giorno in cui le tecnologie informatiche diventeranno necessarie, ossia la base della nostra esistenza, è già arrivato, quindi il giorno in cui si concluderà il \textit{power stage}, sarà anche il giorno che concluderà la \textit{computer revolution}.
Il \textit{power stage} è un periodo che durerà per moltissimo tempo e che vedrà un enorme sviluppo sia a livello scientifico, che a livello etico. 
Dato che l’ambito etico riguardante lo sviluppo delle tecnologie informatiche è davvero vasto e, anno dopo anno, si sta sempre di più ampliando, è stato necessario provvedere a creare una branca specifica dell’etica, chiamata \textit{computer ethics}.
Lo sviluppo di tecnologie informatiche sta diventando sempre più delicato e quindi necessita, senza ombra di dubbio, della \textit{computer ethics}.
Il \textit{power stage}, quindi, non è altro che il periodo della \textit{computer ethics}.
La \textit{computer ethics} non fu presa in considerazione quando lo sviluppo delle tecnologie informatiche era ancora agli albori per il semplice fatto che difficilmente si poteva prevedere sia uno sviluppo così rapido, sia la nascita di così tanti problemi etici derivati dall’evoluzione informatica (si provi ad esempio a pensare a quanti dilemmi sono nati con lo sviluppo dell’intelligenza artificiale).
L’etica dell’informatica non potrà sicuramente rispondere ad ogni domanda, ma sicuramente può fissare dei paletti e descrivere un percorso da seguire per evitare di superare alcuni limiti che non dovrebbero mai essere superati, né ora, né in un futuro in cui potremmo iniziare a pensare come macchine, o in cui le macchine potrebbero iniziare a pensare come noi (ad oggi ci sono differenti scuole di pensiero, inoltre, sulla capacità di pensiero di una macchina. “Le macchine possono pensare? E se la risposta è no, potranno mai pensare?”: queste sono alcune domande di cui la \textit{computer ethics} si occupa).
Si può dunque affermare che lo sviluppo della \textit{computer ethics} è direttamente proporzionale allo sviluppo delle tecnologie informatiche.
Si può quindi anche affermare che argomenti etici affrontati in passato possano subire delle modifiche ed essere riadattati a situazioni più attuali? Lo scopo di questo documento è proprio rispondere a questa domanda, in particolare prendendo come riferimento quanto detto da Moor e utilizzando come esempio il suo \textit{fattore di invisibilità} (Moor 1985).