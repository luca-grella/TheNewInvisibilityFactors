\phantomsection
\subsection*{L’informazione odierna: dall’importanza della tecnologia alla gestione dei dati}
\addcontentsline{toc}{subsection}{L’informazione odierna: dall’importanza della tecnologia alla gestione dei dati}

La tecnologia, come detto poco fa, ha iniziato a diventare importante a partire dall’inizio del \textit{permeation stage}, ma ora si può affermare essere diventata parte fondamentale e necessaria della nostra esistenza.
Un esempio pratico è la dipendenza che la maggior parte delle persone oggi ha nei confronti del proprio smartphone. All’interno di esso è praticamente racchiusa la nostra vita: media, contatti, conversazioni e dati personali riguardanti qualsiasi ambito, ma anche dati di cui a volte siamo inconsapevoli o consapevoli solo per metà: si provi a pensare ad esempio alla cronologia della localizzazione e quindi anche alle posizioni visitate più di frequente o addirittura alla nostra posizione attuale (ma di questo ne parlerò più avanti). Perdere uno smartphone senza backup sarebbe, in ogni caso, molto scomodo e, potendo prevederne lo smarrimento o, in generale, una futura possibilità di inutilizzo, chiunque prenderebbe provvedimenti in anticipo per evitare di perdere dati o rimanere senza smartphone. 
Questa precisazione è utile per introdurre un esempio a sostegno della mia tesi. 
L’importanza attuale della tecnologia si può notare anche a livello sociopolitico o economico. Perché? “Il 16 maggio il Presidente degli Usa Donald Trump ha firmato un documento che vieta a Huawei, di vendere ed installare le proprie infrastrutture negli Stati Uniti senza una specifica autorizzazione” \textit{(www.focus.it)}. Sebbene sia stata una mossa politica per favorire gli Usa (in particolare le aziende statunitensi in contrapposizione a quelle cinesi), è molto interessante soffermarsi, non tanto sul “perché”, ma quanto sul “come”. Come posso colpire maggiormente una persona? Togliendogli ciò di cui non può fare a meno e, in questo caso, è proprio ciò che è avvenuto: il sistema operativo firmato Google smetterà di supportare i dispositivi Huawei a partire da agosto, mettendo in difficoltà gli utenti Huawei americani e non, che probabilmente passeranno ad altri marchi in seguito alla fidelizzazione avvenuta, anno dopo anno, dal sistema operativo di Google, che invece continuerà a funzionare su tutti gli altri dispositivi.
Diventa molto interessante pensare a quali sarebbero le conseguenze se il blocco Usa a Huawei o altre aziende cinesi continuasse: gli utenti Huawei dovrebbero cambiare dispositivo (piuttosto che farne a meno) e questo sottolinea ancora una volta l’enorme dipendenza dalle tecnologie informatiche che si è espansa esponenzialmente negli ultimi anni.
Due problemi che sono la diretta conseguenza di questa dipendenza (che ho ridotto all’uso dello smartphone, ma che vale per numerosi altri esempi) sono: la gestione dei nostri dati che abbiamo volontariamente condiviso e, come ho accennato prima, la creazione autonoma di dati di cui siamo all’oscuro o consapevoli solo per metà (per esempio il \textit{tracking} delle nostre posizioni).
In entrambi i casi la gestione dei dati è in mano a dei singoli colossi: Facebook, Google…
La particolarità di questa situazione è che ognuno di noi è consapevole del fatto che qualsiasi dato condiviso potrebbe non essere in buone mani (vedi lo scandalo Cambridge Analytica che ha coinvolto Zuckerberg e Facebook), ma in fin dei conti, nella maggior parte dei casi, la cosa non ci tange. 
Oggi condividiamo tutto e un gestore del servizio che utilizziamo, ipoteticamente parlando, potrebbe sapere qualsiasi cosa di noi, ma noi viceversa abbiamo accessi molto ridotti e vincolati, spesso anche ai nostri stessi dati o ai dati di un nostro parente stretto. 
Un esempio è il caso di Apple del 2016. Un padre, dopo la precoce morte del figlio ancora 13enne, ha richiesto l’accesso ai dati del suo smartphone, dimostrando che l’accesso gli veniva consentito dal figlio stesso quando era ancora in vita (a dimostrazione l’impronta digitale del padre salvata sul telefono, non più utilizzabile dopo 48 ore di inattività o dopo lo spegnimento poiché sostituita da un codice di blocco, in questo caso non conosciuto dal padre). Accesso che non è stato permesso da Apple per varie ragioni, sia di natura legale, sia di natura morale. Ora la questione è molto delicata: eticamente, moralmente e umanamente potrebbe sembrare più che lecita la richiesta del padre, ma bisogna prendere in considerazione anche altri aspetti che, sul momento e a caldo potrebbero passare in secondo piano. Per esempio “Cook precisò che forzare il codice criptato del cellulare, avrebbe costituito un precedente pericoloso. Questa è la realtà. Aprire una \textit{backdoor} in quel telefono o in mille altri, significa minare la sicurezza delle tecnologie elaborate appositamente per proteggere i dati degli utenti, dati che possono essere più o meno sensibili, ma che rimangono comunque personali. Non è un caso se, in tal senso, molte piattaforme, come Facebook e Google, hanno reso disponibile una funzione che rappresenta un testamento digitale volontario dell’utente affinché egli possa assicurare l’accesso ai propri profili a persone puntualmente individuate” \textit{(www.agi.it)}.
Si può notare che ogni esempio sopra riportato fa sempre riferimento alla gestione dei dati. La gestione dei dati, a volte, non è però solo un servizio offerto al cliente che porta benefici a livello di ritorno economico, ma è una vera e propria fonte di interesse per il servizio stesso che la utilizza per profilare e creare un identikit vero e proprio dell’utente (questo identikit viene effettuato non solo grazie ai dati offerti liberamente dall’utente, ma anche dalla sua predisposizione a certi post, alle sue condivisioni e al suo tempo passato a utilizzare il servizio. Per esempio, il \textit{profiling} potrebbe permettere di conoscere l’orientamento politico e religioso dell’utente, senza che, paradossalmente, egli l’abbia mai dichiarato pubblicamente o scritto da qualsivoglia parte). Unendo e triangolando i dati della \textit{profilazione} di milioni di persone si possono ottenere informazioni riguardo a qualsiasi cosa e si possono effettuare delle indagini su scala globale. Spesso il \textit{profiling} è legale, poiché i colossi che gestiscono i dati di milioni di utenti si muovono all’interno dei confini dettati dalla legge (a parti rari casi, come per esempio il già citato scandalo Cambridge Analytica), ma è eticamente giusto?