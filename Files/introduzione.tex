Negli ultimi decenni, il mondo dell’informatica e delle tecnologie dell’informazione sta assumendo sempre più importanza. Con esso si stanno sviluppando a pari passo molti problemi di natura etica. 
In questo documento affronterò un discorso riguardante lo sviluppo dell’etica, sottolineando la sua importanza nel contesto informatico, soffermandomi su alcuni esempi e particolarità appartenenti al campo dell’informatica odierna.
Nello specifico il documento è composto da 4 capitoli più un quinto capitolo conclusivo.
Il primo capitolo si occuperà di dare un quadro completo della \textit{computer revolution} sulla base di quanto affermato da Moor in alcuni suoi scritti, descrivendo la suddivisione degli stage (periodi) e soffermandosi sul ruolo della \textit{computer ethics} nell’ultimo stage, ossia quello in cui viviamo.
Entrando nel vivo del documento, il secondo capitolo descriverà il \textit{fattore di invisibilità} di Moor, parte fondamentale della mia tesi essendo, a mio parere, una delle chiavi centrali dello sviluppo della \textit{computer ethics}.
Il terzo capitolo, invece, affronterà il discorso dell’importanza della tecnologia e della gestione dei dati, sempre in chiave etica e sempre riadattandosi al discorso del \textit{fattore di invisibilità} di Moor. Inoltre, saranno mostrati alcuni esempi molto attuali a sostegno della mia tesi.
Infine, nell’ultimo capitolo e nella conclusione, verranno unite tutte queste informazioni per parlare di privacy, di \textit{profilazione} e di come il \textit{profiling} possa, non essendo riconducibile a nessuna delle categorie descritte da Moor, essere definito come nuovo esempio di \textit{fattore di invisibilità}. Punto cardine di questa mia tesi è la necessità dell’ampliamento della definizione di \textit{fattore di invisibilità}.
Tutto ha un punto di partenza e, nel nostro caso, il punto di partenza è proprio ciò che ci ha resi quelli che siamo: l’evoluzione.