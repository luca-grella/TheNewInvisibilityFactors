\phantomsection
\subsection*{Una particolarità delle tecnologie informatiche: il \textit{fattore di invisibilità}}
\addcontentsline{toc}{subsection}{Una particolarità delle tecnologie informatiche: il \textit{fattore di invisibilità}}

Per capire appieno la teoria di Moor riguardante il \textit{fattore di invisibilità} bisogna fare alcune precisazioni. Ad oggi la maggior parte delle operazioni informatiche sono invisibili. Cosa significa “invisibili”? 
Il discorso è molto vasto, ma io mi soffermerò solo su alcuni esempi che, da soli, possono far comprendere meglio il mio pensiero. 
Quando avviene un processo informatizzato la maggior parte degli utenti, ma a volte anche lo stesso programmatore, sono inconsapevoli di buona parte delle elaborazioni interne (se non di tutte). Per questo motivo si può parlare di operazioni informatiche “invisibili”.
Il discorso, così spiegato, può sembrare molto astratto, ma Moor afferma che esistono 3 tipi di invisibilità (Moor 1985) e affianca questo suo pensiero ad alcuni esempi pratici che esporrò. Inoltre, porterò altri esempi, non riconducibili a nessuna delle 3 categorie dei fattori di invisibilità di Moor, per affermare che gli \textit{invisibility factor} possono essere diversi e le categorie ampliate a più di tre. Ovviamente questo è un discorso che può essere fatto a posteriori, in quanto, quando Moor formalizzò la sua teoria sui “fattori di invisibilità”, gli esempi a cui farò riferimento, o non erano attuali, oppure, con le tecnologie di qualche decennio fa, non potevano neanche essere pensati o previsti.
Il primo \textit{fattore di invisibilità} è quello che viene definito “abuso invisibile”. Esso si verifica in situazioni quali, per esempio: furto di informazioni personali in seguito ad un accesso non autorizzato (riconducibile ad una invasione della proprietà privata e della riservatezza altrui) e furto di un interesse in eccesso in banca. Nella maggior parte di questi casi, la vittima non si accorge di essere stata violata nemmeno dopo molto tempo e quindi l’aggettivo “invisibile” assume ancor più valore, nonostante esso possa già essere definito tale in seguito alla difficoltà di scoprire un “abuso invisibile” mentre viene eseguito.
Il secondo \textit{fattore di invisibilità} è costituito dai “valori di programmazione invisibili”, ossia quei valori che sono incorporati in un programma per computer. In questo caso si fa riferimento al rapporto \textit{user-programmer}, in quanto, il più delle volte, il programmatore formula giudizi sul valore di cosa può essere importante o meno per l’utente, decidendo cosa mostrare. Ciò che viene ritenuto poco importante, viene reso invisibile all’esecutore (la maggior parte dei valori incorporati nel programma finale, dunque, sono invisibile all’utente).
Il terzo ed ultimo \textit{fattore di invisibilità} è il “calcolo complesso invisibile”. Potrebbe sembrare il meno importante, ma, se ci si ragiona bene, grazie ad esso si può comprendere quanto ci stiamo affidando, nel corso degli anni, alle tecnologie informatiche. 
Spesso per arrivare a risultati o output di qualsiasi natura e tipo, ci affidiamo a programmi che svolgono operazioni molto complesse. Sono quest’ultime che possono essere ricondotte a calcoli complessi invisibili. Perché? Esse non possono essere analizzate da un essere umano a causa della loro complessità. Se dovessero esserci errori nel processo non potrebbero quindi essere identificati. Per questo motivo ci conviene “fidarci della macchina” e accettare il suo output come corretto senza poterlo verificare. 
Ovviamente tutti questi esempi sono molto attuali e le relative definizioni di \textit{invisibility factor} funzionano anche oggi, ma questa suddivisione in 3 macro-categorie può sembrare molto stretta e insufficiente se applicata ad altri ambiti che si sono sviluppati solo negli ultimi anni.
Per ovviare a questo problema e applicare la teoria di Moor all’informazione odierna bisogna procedere all’espansione dei tre \textit{invisibility factor}.