L’evoluzione umana ha avuto un incremento esponenziale con lo sviluppo della tecnologia. Oggi la tecnologia è una risorsa fondamentale e necessaria. Lo sviluppo tecnologico e informatico ha, però, anche i suoi contro. I nostri dati sono in mano a singoli colossi. Si condivide tutto e un gestore del servizio che utilizziamo potrebbe sapere tutto di noi, ma noi, viceversa, potremmo avere problemi ad accedere ai nostri stessi dati. Tutto questo sviluppo (o inviluppo) tecnologico, aumentando i casi di \textit{invisibility factor} (termine coniato da Moor riferendosi all’invisibilità di tutte quelle operazioni informatiche delle quali si è inconsapevoli in quanto essenti parte integrante, ad esempio, dell’elaborazione interna dei vari software), porta alla necessità di ampliare la \textit{computer ethics}. Un esempio pratico è l’invisibilità della \textit{profilazione}, diretta conseguenza a un’evoluzione delle metodologie di gestione dei dati. Ampliare la \textit{computer ethics}, inoltre, permette l’espansione degli \textit{invisibility factor} per applicare la teoria di Moor all’informazione odierna.