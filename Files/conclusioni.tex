Per concludere possiamo affermare quindi che il \textit{profiling} non è altro che un esempio diretto che può portare a un dibattito etico riguardante gli standard odierni sulla privacy. Inoltre, essendo direttamente coinvolto in questo discorso ho provato a portare una mia tesi riprendendo il pensiero di alcuni filosofi che hanno approfondito il discorso della \textit{computer ethics} negli ultimi anni. Le questioni di etica e legalità sono essenziali in molti settori. 
Medici, insegnanti, funzionari governativi e uomini d'affari hanno tutti una supervisione legale ed etica per controllare come funzionano le loro professioni. La tecnologia dell'informazione, al contrario, non ha una standardizzazione generale in atto. Tuttavia, poiché la tecnologia dell'informazione diventa sempre più influente, le considerazioni etiche e legali diventano altrettanto pertinenti. La maggior parte delle persone ha i propri dati personali diffusi in tutto il mondo digitale. Anche le cose ritenute sicure, come account di posta elettronica o privati, sono accessibili da fonti non intenzionali. La maggior parte dei datori di lavoro controlla attivamente le abitudini dei loro dipendenti. La privacy ha implicazioni legali in evoluzione, ma ci sono anche considerazioni etiche. Le persone sanno come vengono monitorati i loro account? In che misura si verifica tale monitoraggio? I problemi di privacy possono facilmente diventare una china scivolosa, erodendo lentamente e completamente il diritto alla privacy di un individuo. 
I media digitali hanno permesso alle informazioni di fluire più liberamente di prima. Questo scambio di idee ha una reazione legale ed etica. Come si può stabilire la proprietà nel regno digitale? Le cose possono essere facilmente copiate e incollate online, il che rende difficile il controllo della proprietà intellettuale. Le nozioni legali come il copyright hanno faticato a tenere il passo con l'era digitale. Le aziende nel settore della musica e dell'intrattenimento hanno spinto per maggiori protezioni legali per le proprietà intellettuali, mentre altri attivisti hanno cercato di fornire maggiori libertà per lo scambio di idee nel regno digitale.
A un certo livello, tutti sanno che le loro vite online sono monitorate. Gli Stati Uniti hanno persino approvato una legge che consente al governo di monitorare attivamente i cittadini privati in nome della sicurezza nazionale. Queste misure hanno riacceso un dibattito su quali informazioni possono essere raccolte e perché. Questo dibattito si applica anche su scala ridotta, perché le aziende devono considerare quali informazioni raccogliere dai propri dipendenti. Questo problema richiama due domande: le persone sanno quali informazioni vengono monitorate? Hanno il diritto di sapere come vengono utilizzati i loro dati?
In passato, i problemi di sicurezza venivano risolti chiudendo una porta. La sicurezza digitale è molto più complicata. I sistemi di sicurezza per le reti digitali sono informatizzati per proteggere le informazioni vitali e le risorse importanti. Tuttavia, questa maggiore sicurezza è accompagnata da una maggiore sorveglianza. Tutti i sistemi di sicurezza hanno rischi intrinseci, il che significa che si tratta di quali rischi sono accettabili e quali libertà possono essere perse. In definitiva, i professionisti delle tecnologie informatiche devono bilanciare il rischio con la libertà di creare un sistema di sicurezza che sia efficace ed etico allo stesso tempo.
La neutralità della rete è diventata una questione di tendenza grazie agli sforzi legislativi negli ultimi anni. La questione della neutralità della rete è essenzialmente una questione di accesso. I fautori vogliono che Internet rimanga aperto a tutti mentre alcune aziende vogliono creare un accesso a più livelli per coloro che sono disposti a pagare. Il problema si estende anche all'uso privato di Internet poiché il costo del servizio in alcune aree potrebbe essere proibitivo. La più ampia questione etica è se lo scambio digitale sia o meno un diritto universale. Il costo dell'accesso può ostacolare la crescita del business, lo spirito imprenditoriale e l'espressione individuale.
Questi problemi sono essenziali per tutti, ma hanno un peso extra per coloro che lavorano con le tecnologie dell'informazione. È importante ricordare che lavorare con la tecnologia non è separato dai contesti etici, ma può effettivamente aiutare a definire un codice legale ed etico per le generazioni a venire.
Ho voluto ampliare il \textit{profiling} come discorso pratico, per portare un esempio di come gli \textit{invisibility factor} di Moor siano oggi insufficienti. Inoltre, tutti gli esempi riguardanti il discorso della privacy sopracitati, possono essere letti in chiave di \textit{invisibility factor}. Tutto ciò che è futuro, può essere letto e interpretato in chiave passata e la privacy non si discosta da questo discorso: \textit{“Privacy is in modern societies an ideal rooted in the Enlightenment”} (Fuchs 2010).
È bello poter pensare al futuro come qualcosa che viene costruito, giorno dopo giorno, su solide basi. Quelle basi dipendono dal presente, dipendono dal nostro pensiero e dalle nostre idee, dipendono da noi. Tutto ciò che noi siamo lo dobbiamo a ciò che ci è stato trasmesso dal passato. Tutto ciò che è passato, in chiave riadattata, può essere applicato a ciò che ci circonda oggi o a ciò che verrà: spero di averlo trasmesso con questo breve documento, esprimendo quanto, un’idea di più di 30 anni fa, possa essere così attuale.